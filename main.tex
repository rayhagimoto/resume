%==========================================================
% template.tex
%==========================================================
\documentclass[letterpaper,10pt]{article}
\usepackage[utf8]{inputenc}
\usepackage[top=0.5in,left=0.5in,right=0.5in,bottom=0.5in]{geometry}
\usepackage{enumitem}
\usepackage[hidelinks]{hyperref}
\usepackage{xstring}

%-----------------------------------------------------
% Strip off “<scheme>://” from a URL, leaving just the
% domain+path.  We look for “://” and grab everything
% after it.  If there is no “://”, we output the whole
% string unchanged.
%-----------------------------------------------------
\newcommand{\ShortURL}[1]{%
  \IfSubStr{#1}{://}{%
    % if “://” appears, take everything after “://”
    \StrBehind{#1}{://}[\urlbody]%
    \urlbody
  }{%
    % otherwise, no “://” → just print #1
    #1
  }%
}


%----------------------------------------------------------
% Overall spacing (no indent, no extra paragraph skip):
%----------------------------------------------------------
\setlength{\parindent}{0pt}
\setlength{\parskip}{0pt}

%----------------------------------------------------------
% itemize: zero extra space before/after and minimal between items
%----------------------------------------------------------
\setlist[itemize]{%
  topsep=0pt,     % no extra gap above/below list
  itemsep=0pt,    % no gap between successive \item
  parsep=0pt,     % no gap between paragraphs inside an item
  partopsep=0pt,  % no “part‐paragraph” space if list is mid‐paragraph
  leftmargin=1em  % indent for bullets
}

%----------------------------------------------------------
% A tiny vertical gap macro (for “section” separation)
%----------------------------------------------------------
\newcommand{\sectionspace}{\vspace{0.5em}}

%----------------------------------------------------------
% Remove page numbers
%----------------------------------------------------------
\pagenumbering{gobble}

%----------------------------------------------------------
% === PERSONAL INFO SLOTS ===
%   These macros will be overridden in content.tex
%----------------------------------------------------------
\newcommand{\Name}{}
\newcommand{\Title}{}
\newcommand{\Phone}{}
\newcommand{\Email}{}
\newcommand{\Location}{}
\newcommand{\LinkedInURL}{}
\newcommand{\WebsiteURL}{}

%----------------------------------------------------------
% === HEADER MACRO ===
%   Prints Name, Title, and Contact line
%----------------------------------------------------------
\newcommand{\MakeHeader}{%
  \begin{center}
    {\large\bfseries \Name}\\
    {\itshape \Title}\\
    \sectionspace%
    \Phone \,$|$\, 
    \href{mailto:\Email}{\Email} \,$|$\, 
    \Location \,$|$\, 
    \href{\LinkedInURL}{\ShortURL{\LinkedInURL}} \,$|$\, 
    \href{\WebsiteURL}{\ShortURL{\WebsiteURL}}
  \end{center}
  \sectionspace
}

%----------------------------------------------------------
% === PROFILE SLOT ===
%   Overridden to hold a short paragraph in content.tex
%----------------------------------------------------------
% A custom “horizontal rule” that always leaves 0.5em below it:
\newcommand{\SectionRule}{%
  \par\noindent
  \hrule width \textwidth height 0.5pt
  \vspace{0.5em}%
}
\newcommand{\ProfileText}{}

%----------------------------------------------------------
% Helper  →  Bold everything up to the FIRST comma
%----------------------------------------------------------
\newcommand{\BoldBeforeComma}[1]{%
  % Append a “,\relax” so that we always have a comma to split on
  \expandafter\BoldBeforeCommaAux#1,\relax
}

% This auxiliary macro grabs “<text before comma>” as #1
% and “<text after comma>” as #2, stopping at the \relax.
\def\BoldBeforeCommaAux#1,#2\relax{%
  {\bfseries #1}%        % bold the part before the comma
  \ifx\relax#2\relax     % if #2 == \relax, there was no comma → do nothing
  \else
    ,#2                   % otherwise reprint the comma and whatever follows
  \fi
}

%----------------------------------------------------------
% === MACROS FOR EDUCATION ENTRIES ===
%   #1 = Institution + City/State
%   #2 = Dates
%   #3 = “Degree (GPA: …)”
%   #4 = <\begin{itemize} … \end{itemize}> block
%----------------------------------------------------------
\newcommand{\EducationEntry}[4]{%
  \noindent
  \begin{tabular*}{\textwidth}{l@{\extracolsep{\fill}}r}
    \BoldBeforeComma{#1} & #2 \\        % <— use the helper here
    \multicolumn{2}{l}{\textit{#3}} \\
  \end{tabular*}%
  #4%
  \sectionspace
}

%----------------------------------------------------------
% === MACROS FOR EXPERIENCE ENTRIES ===
%   #1 = Company + City/State
%   #2 = Dates
%   #3 = Job Title
%   #4 = <\begin{itemize} … \end{itemize}> block
%----------------------------------------------------------
\newcommand{\ExperienceEntry}[4]{%
  \noindent
  \begin{tabular*}{\textwidth}{l@{\extracolsep{\fill}}r}
    \BoldBeforeComma{#1} & #2 \\        % <— use the helper here
    \multicolumn{2}{l}{\textit{#3}} \\
  \end{tabular*}%
  #4%
  \sectionspace
}

%----------------------------------------------------------
% === MACROS FOR LEADERSHIP & COMMUNITY ENTRIES ===
%   #1 = Organization + City/State
%   #2 = Dates
%   #3 = Role
%   #4 = <\begin{itemize} … \end{itemize}> block
%----------------------------------------------------------
\newcommand{\LeadershipEntry}[4]{%
  \noindent
  \begin{tabular*}{\textwidth}{l@{\extracolsep{\fill}}r}
    \BoldBeforeComma{#1} & #2 \\        % <— use the helper here
    \multicolumn{2}{l}{\textit{#3}} \\
  \end{tabular*}%
  #4%
  \sectionspace
}

%----------------------------------------------------------
% === MACRO FOR AWARDS ===
%   #1 = Award‐Name (Year)
%   This is used inside a single \begin{itemize} … \end{itemize}.
%----------------------------------------------------------
\newcommand{\AwardEntry}[1]{%
  \item #1
}

%==========================================================
% Document begins.  First we “\input” content.tex so that
%   all personal macros (\Name, \ProfileText, etc.) are defined.
%==========================================================
\begin{document}

% 0) Load the user’s content (overrides all macros and defines
%    grouped‐entry commands like \EducationEntries, \ExperienceEntries, etc.)
%==========================================================
% content.tex
%
% This file:
%  1. Overrides the “slot” macros (\Name, \Phone, \ProfileText, etc.)
%  2. Defines grouped commands (\EducationEntries, \ExperienceEntries, etc.)
%     so that the template can call them in the right order.
%
% Whenever you update your personal info, or add/remove an entry,
% you modify _only_ this file.  Never touch template.tex.
%==========================================================

%----------------------------------------------------------
% 1) PERSONAL INFO OVERRIDES
%----------------------------------------------------------
\renewcommand{\Name}{RAY HAGIMOTO}
\renewcommand{\Title}{Computational Physicist}
\renewcommand{\Phone}{830-212-9632}
\renewcommand{\Email}{rayhagimoto@gmail.com}
\renewcommand{\Location}{Danville, IL}
\renewcommand{\LinkedInURL}{https://linkedin.com/in/ray-hagimoto}
\renewcommand{\WebsiteURL}{https://rayhagimoto.xyz}

%----------------------------------------------------------
% 2) PROFILE TEXT OVERRIDE
%----------------------------------------------------------
\renewcommand{\ProfileText}{%
Computational physics PhD with 5 years of data analysis and Python programming experience. Seeking quantitatively complex challenges in data science. Experienced with Monte Carlo simulations and statistical inference on real data.%
}

%----------------------------------------------------------
% 3) EDUCATION ENTRIES
%    We collect all \EducationEntry calls into one macro:
%----------------------------------------------------------
\newcommand{\EducationEntries}{%
  %— Rice University (PhD)
  \EducationEntry{%
    Rice University, Houston, TX%
  }{%
    Aug 2020 -- Dec 2024%
  }{%
    Doctor of Philosophy, Physics (GPA: 3.90)%
  }{%
    \begin{itemize}
      \item Coursework includes computational physics (Python), probability theory, and quantum field theory.
    \end{itemize}
  }
  %— UT San Antonio (BS)
  \EducationEntry{%
    University of Texas at San Antonio, San Antonio, TX%
  }{%
    Aug 2016 -- May 2020%
  }{%
    Bachelor of Science, Physics (GPA: 4.00)%
  }{%
    \begin{itemize}
      \item Coursework includes mathematics (linear algebra; calculus I, II, III) and advanced physics.
    \end{itemize}
  }
}

%----------------------------------------------------------
% 4) SKILLS & INTERESTS
%    Define \SkillsText as a single paragraph (no extra \sectionspace)
%----------------------------------------------------------
\newcommand{\SkillsText}{%
\textbf{Skills:} Python, Jupyter, git, UNIX/Linux, NumPy, Pandas, LightGBM, SHAP, scikit-learn, machine learning, HDF5, matplotlib, HPC and cloud computing, SLURM%
}

%----------------------------------------------------------
% 5) EXPERIENCE ENTRIES
%    Collect all \ExperienceEntry calls into \ExperienceEntries
%----------------------------------------------------------
\newcommand{\ExperienceEntries}{%
  %— Susquehanna Intern
  \ExperienceEntry{%
    Susquehanna International Group, Bala Cynwyd, PA%
  }{%
    June 2024 -- Aug 2024%
  }{%
    Quantitative Researcher PhD Intern%
  }{%
    \begin{itemize}
      \item Participated in 10-week quantitative finance internship at the largest US equity options market-making firm.
      \item Performed data analysis on financial time series data and developed a trading strategy in a simulated environment.
      \item Used Shapley values to aid in feature engineering and trained a boosted decision tree model for my trading signal.
    \end{itemize}
  }
  %— Rice University (Graduate Researcher)
  \ExperienceEntry{%
    Rice University, Houston, TX%
  }{%
    Aug 2020 -- Dec 2024%
  }{%
    Graduate Researcher%
  }{%
    \begin{itemize}
      \item Trained convolutional neural networks to perform parameter estimation on cosmological data.
      \item Created a Python package to find statistical features in astrophysical data sets using Bayesian methods and Markov Chain Monte Carlo simulations.
      \item Wrote custom scripts for data visualization by extending and adapting ArviZ source code.
    \end{itemize}
  }
  %— University of Chicago (Undergrad Researcher)
  \ExperienceEntry{%
    University of Chicago, Chicago, IL%
  }{%
    June 2019 -- Sept 2019%
  }{%
    Undergraduate Summer Researcher%
  }{%
    \begin{itemize}
      \item Wrote a Python program to calculate two-point correlation functions for an astrophysical data set.
      \item Published team’s results in \textit{The Astrophysical Journal Letters} and presented findings at multiple conferences.
    \end{itemize}
  }
}

%----------------------------------------------------------
% 6) LEADERSHIP & COMMUNITY INVOLVEMENT
%    Collect all \LeadershipEntry calls into \LeadershipEntries
%----------------------------------------------------------
\newcommand{\LeadershipEntries}{%
  \LeadershipEntry{%
    Physics Graduate Student Association, Rice University, Houston, TX%
  }{%
    2022 -- 2023%
  }{%
    Vice President%
  }{%
    \begin{itemize}
      \item Organized professional development workshops and social events.
      \item Liaised closely with other officers, professors, and external organizations.
      \item Mentored incoming graduate students.
    \end{itemize}
  }
  \LeadershipEntry{%
    Society of Physics Students, UTSA, San Antonio, TX%
  }{%
    2019 -- 2020%
  }{%
    Vice President%
  }{%
    \begin{itemize}
      \item Organized club meetings, fundraising activities, and social events.
      \item Created a mentor‐mentee program to improve support for undergraduate students.
    \end{itemize}
  }
}

%----------------------------------------------------------
% 7) SELECTED HONORS & AWARDS
%    Define \AwardsList as a single itemize environment
%----------------------------------------------------------
\newcommand{\AwardsList}{%
  \AwardEntry{NSF Graduate Research Fellowship Program Honorable Mention 2021}%
  \AwardEntry{Rice University Dean’s Fellowship 2020}%
}

%==========================================================
% End of content.tex
%==========================================================


% 1) Header
\MakeHeader

% 2) PROFILE
\noindent\textbf{PROFILE}\newline
\SectionRule
\ProfileText    % <— content.tex must have redefined this
\sectionspace

% 3) EDUCATION
\noindent\textbf{EDUCATION}\newline
\SectionRule
\sectionspace
\EducationEntries   % <— content.tex must define this

% 4) SKILLS & INTERESTS
\sectionspace
\SkillsText         % <— content.tex must define this
\sectionspace

% 5) EXPERIENCE
\noindent\textbf{EXPERIENCE}\newline
\SectionRule
\sectionspace
\ExperienceEntries  % <— content.tex must define this

% 6) LEADERSHIP & COMMUNITY INVOLVEMENT
\noindent\textbf{LEADERSHIP \& COMMUNITY INVOLVEMENT}\newline
\SectionRule
\sectionspace
\LeadershipEntries  % <— content.tex must define this

% 7) SELECTED HONORS AND AWARDS
\noindent\textbf{SELECTED HONORS AND AWARDS}\newline
\SectionRule
\sectionspace
\begin{itemize}
  \AwardsList      % <— content.tex must define this
\end{itemize}

\end{document}
